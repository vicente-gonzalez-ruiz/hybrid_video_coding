\title{Motion Compensation}
%\title{Hybrid Video Coding}
\author{Vicente González Ruiz}
\maketitle
\tableofcontents

\section{Decision mode and RDO}
%{{{ 

Most video coding standards operate at the block level. For each
block, the decision mode (run only by the compressor) determines the
``type of block'':

\begin{enumerate}
\item \textbf{I-type (intra)}, when the block is encoded without
  considering any other external reference information. In this case,
  no resudue is generated), and a I-type block represents only
  texture. RDO selects this mode when the compression of residue block
  generates more bits than the original (predicted) one.
\item \textbf{P-type (predicted)}, if there is one reference (block)
  for the (predicted) block, which belongs to a previous frame. Now we
  encode a block with residual texture and a motion vector per
  block. P-type blocks require less data than I-type blocks (this is
  guaranteed by the RDO).
\item \textbf{B-type (bidirectionally-predicted)}, if there are two or
  more references for the block, anterior (past) and posterior
  (future). By definition (RDO), it is more RD-advantageous to use
  B-type blocks compared to P-type blocks, because the entropy of the
  residual texture decreases (even considering that now we need one
  motion vector per reference).
\item \textbf{S-type (skipped)}, if the residue is so small that it is
  more beneficial to consider that it is zero. Consequently, the block
  is a copy of the reference.
\end{enumerate}

\begin{figure}
  \centering
  \myfig{graphics/macroblocks}{6cm}{600}
  \caption{Types of macro-blocks.}
  \label{fig:macroblocks}
\end{figure}

The type of the blocks is signaled in the code-stream. An visual
example of the decision of the type of the macro-blocks is shown in
the Figure~\ref{fig:macroblocks}.

%}}}
\section{Hybrid video coding}
%{{{ 

\begin{figure}
  \svg{graphics/hybrid_coding}{800}
  \caption{Hybrid video coding.}
  \label{fig:hybrid_coding}
\end{figure}

See the Fig.~\ref{fig:hybrid_coding}.

\begin{itemize}
\tightlist
\item
  Used in all the video compression standards.
\item
  Intracoded images are transformed ($T$) and quantized ($Q$).
\item
  Intercoded images are also motion compensated ($P$).
\item
  The encoder incorporates a decoder to avoid the drift error.
\end{itemize}

% }}}

\subsection{Codec}
\begin{figure}
  \centering
    \svg{graphics/codec}{1100}
  \caption{A inter/intra video codec.}
\label{fig:IPP_codec}
\end{figure}

The Figure~\ref{fig:IPP_codec} describes an inter/intra video codec,
that corresponds to:

\begin{equation}
  {\mathbf W}_k = \text{color-T}({\mathbf V}_k),
  \tag{a}
\end{equation}

\begin{equation}
   {\mathbf W}_{k-1} = Z^{-1}({\mathbf W}, k-1),
  \tag{b}
\end{equation}
and by definition, $Z^{-1}({\mathbf W}, -1) = {\mathbf 0}$,

\begin{equation}
  \overset{(k-1)\rightarrow k}{\mathbf M} = \text{ME}({\mathbf W}_{k-1}, {\mathbf W}_k),
  \tag{c}
\end{equation}
where ME stands for Motion Estimation, and by definition,
$\overset{(-1)\rightarrow 0}{{\mathbf M}} = {\mathbf 0}$,

\begin{equation}
  \overset{(k-1)\rightarrow k}{\mathbf M} = \overset{(k-1)\rightarrow k}{\mathbf M} \text{(lossless~coding)},
  \tag{d}
\end{equation}

\begin{equation}
  \overset{(k-1)\rightarrow k}{\mathbf M} = \overset{(k-1)\rightarrow k}{\mathbf M} \text{(lossless~decoding)},
  \tag{e}
\end{equation}

\begin{equation}
  {\mathbf E}_k = {\mathbf W}_k - \overset{\wedge}{{\mathbf W}}_k,
  \tag{f}
\end{equation}
where the symbol $-$ represents to the pixel-wise substraction,

\begin{equation}
  \overset{\sim}{{\mathbf E}_k} = \text{QE}({\mathbf E}_k),
  \tag{g}
\end{equation}
where QE$(\cdot)$ represents the lossy compression of the
prediction error texture data,

\begin{equation}
  \overset{\sim}{\mathbf E}_k = \text{DQ}^{-1}(\overset{\sim}{\mathbf E}_k),
  \tag{h}
\end{equation}
where DQ$^{-1}(\cdot)$ represents the decompression of the prediction
error texture data,

\begin{equation}
  \overset{\sim}{\mathbf W}_k \leftarrow \overset{\sim}{\mathbf E}_k + \overset{\wedge}{\mathbf W}_k,
  \tag{i}
\end{equation}
and notice that if $\overset{\wedge}{\mathbf W}_k = {\mathbf 0}$, then
$\overset{\sim}{\mathbf E}_k = \overset{\sim}{\mathbf W}_k$,

\begin{equation}
  \overset{\wedge}{\mathbf W}_k = \text{P}(\overset{\sim}{\mathbf W}_{k-1}, \overset{(k-1)\rightarrow k}{\mathbf M}),
  \tag{j}
\end{equation}
where P$(\cdot,\cdot)$ is a motion compensated predictor, and

\begin{equation}
   \overset{\wedge}{\mathbf W}_{k-1} = Z^{-1}(\overset{\wedge}{\mathbf W}, k-1),
  \tag{k}
\end{equation}
where by definition, $Z^{-1}(\overset{\wedge}{\mathbf W}, -1) = 0$, and

\begin{equation}
   {\mathbf V} = \text{color-T}^{-1}({\mathbf W}_k),
  \tag{l}
\end{equation}
is the inverse color transform.

Notice that if $\overset{\wedge}{{\mathbf W}}_k$ is similar to
${\mathbf W}_k$, then ${\mathbf E}_k$ will be approximately zero, and
therefore, easely compressed. Another interesting aspect to highlight
is that the encoder replicates de decoder in order to use the
reconstructed images as reference and avoid the drift error.

\subsection{A block diagram of the step codec}

It's time to test the performance of the ME/MC process previously
described, in the image domain. We encode a sequence of
frames $\{W_k\}$ using the pattern IPP... which means that the first
frame will be intra-coded (I-type frame) and the rest of frames of the
GOF (Group Of
(\href{https://en.wikipedia.org/wiki/Group_of_pictures}{Frames}) will
be predicted-coded (P-type frame), respect to the previous one.

\begin{figure}
  \centering
    \svg{graphics/codec2}{1100}
  \caption{A simple IPP... image codec.}
\label{fig:IPP_codec}
\end{figure}

The IPP... coding can be done by the codec shown in the
Fig.~\ref{fig:IPP_codec}, where:

\begin{equation}
  V_k \leftarrow \text{C}(W_k) =
  \begin{bmatrix}
    \frac{1}{4} &  \frac{1}{2}  &  \frac{1}{4} \\ 
    \frac{1}{2} &            0  & -\frac{1}{2} \\
    -\frac{1}{4} &  \frac{1}{2}  & -\frac{1}{4}
  \end{bmatrix}
  \begin{bmatrix}
    W_k.\text{R} \\
    W_k.\text{G} \\
    W_k.\text{B}
  \end{bmatrix}
  , \tag{a}
\end{equation}

\begin{equation}
  Z^{-1}(V_k) = V_{k-1},
  \tag{b}
\end{equation}
and by definition, $Z^{-1}(V_{-1}) = 0$,

\begin{equation}
  \overset{k\rightarrow k-1}{V} \leftarrow \text{M}(V_k, V_{k-1}),
  \tag{c}
\end{equation}
where M stands for Motion Estimation, and by definition,
$\overset{0\rightarrow (-1)}{V}=0$,

\begin{equation}
  \overset{\sim}{\overset{k\rightarrow k-1}{V}} \leftarrow \text{E}_{\overset{\rightarrow}{V}}(\overset{k\rightarrow k-1}{V}),
  \tag{d}
\end{equation}
where E$_{\overset{\rightarrow}{V}}(\cdot)$ represents the lossy
  compression of the motion data,

\begin{equation}
  \overset{\sim}{\overset{k\rightarrow k-1}{V}} \leftarrow \text{D}_{\overset{\rightarrow}{V}}(\overset{\sim}{\overset{k\rightarrow k-1}{V}}),
  \tag{e}
\end{equation}
where D$_{\overset{\rightarrow}{V}}(\cdot)$ represents the 
decompression of the motion data,

\begin{equation}
  E_k \leftarrow V_k - \overset{\wedge}{{V}}_k,
  \tag{f}
\end{equation}
where the symbol $-$ represents to the pixel-wise substraction,

\begin{equation}
  \overset{\sim}{E_k} \leftarrow \text{E}_{E}(E_k),
  \tag{g}
\end{equation}
where E$_{E}(\cdot)$ represents the lossy compression of the
prediction error texture data,

\begin{equation}
  \overset{\sim}{E}_k \leftarrow \text{D}_{E}(\overset{\sim}{E}_k),
  \tag{h}
\end{equation}
where D$_{E}(\cdot)$ represents the decompression of the prediction
error texture data,

\begin{equation}
  \overset{\sim}{V}_k \leftarrow \overset{\sim}{E}_k + \overset{\wedge}{V}_k,
  \tag{i}
\end{equation}
and notice that if $\overset{\wedge}{V}_k=0$, then
$\overset{\sim}{E}_k = \overset{\sim}{V}_k$,

\begin{equation}
  \overset{\wedge}{V}_k \leftarrow \text{P}(\overset{\sim}{\overset{k\rightarrow k-1}{V}}, \overset{\sim}{V}_{k-1}),
  \tag{j}
\end{equation}
where P$(\cdot,\cdot)$ is a motion compensated predictor.

Notice that if $\overset{\wedge}{{V}}_k$ is similar to $V_k$, then
$E_k$ will be approximately zero, and therefore, easely
compressed. Another interesting aspect to highlight is that the
encoder replicates de decoder in order to use the reconstructed images
as reference and avoid the drift error.

\section{The GOP (Group Of Pictures) concept}
\begin{itemize}
\tightlist
\item
  The temporal redundancy is exploited by blocks of images called GOPs.
  This means that a GOP can be decoded independently of the rest of
  GOPs (see the Fig.~\ref{fig:GOPs}).
\end{itemize}

\begin{figure}
  \svg{graphics/GOPs}{500}
  \caption{A GOP.}
  \label{fig:GOPs}
\end{figure}

\section{Hybrid coding alternatives}
\begin{itemize}
\item
  \textbf{t+2d}: The sequence of images is decorrelated first along the
  time (t) and the residue images are compressed, exploiting the
  remaining spatial (2d) redundancy. Examples: MPEG* and H.26* codecs
  (except H.264/SVC).
\item
  \textbf{2d+t}: The spatial (2d) redudancy is explited first (using
  typically the DWT) and after that, the coefficients are decorrelated
  along the time (t). For now, this has only been an experimental setup
  because most DWT transformed domains are not invariant to the
  displacement, and therefore, ME/MC can not be directly applied.
\item
  \textbf{2d+t+2d}: The fist step creates a Laplacian Pyramid (2d),
  which is invariant to the displacement. Next, each level of the
  pyramid is decorrelated along the time (t) and finally, the
  remaining spatial redundancy is removed (2d). The
  Fig~\ref{fig:H264-S-SVC} show an example for H.264/SVC.
\end{itemize}

\begin{figure}
  \svg{graphics/H264-S-SVC}{500}
  \caption{SVC scheme in H.264.}
  \label{fig:H264-S-SVC}
\end{figure}

\section{Deblocking filtering}
\begin{itemize}
\tightlist
\item
  If any other block-overlaping techniques have not been applied,
  block-based video encoders improve their performance if a deblocking
  filter in used to create the quantized prediction predictions (see
  the Fig.~\ref{fig:350px-Deblock1}).
\end{itemize}

\begin{figure}
  \jpg{350px-Deblock1}{800}
  \caption{Deblocking filetering effect.}
  \label{fig:350px-Deblock1}
\end{figure}

\begin{itemize}
\tightlist
\item
  The low-pass filter is applied only on the block boundaries.
\end{itemize}

\section{Bit-rate allocation}
\begin{itemize}
\item
  VBR: Under a constant quantization level (constant video quality),
  the number of bits that each compressed image needs depends on the
  image content (Variable Bit-Rate). In the
  Fig.~\ref{fig:closed-loop-1_ir} there is an
  example.
  \begin{figure}
    \svg{graphics/closed-loop-1_ir}{500}
    \caption{Example of variable bit-allocation.}
    \label{fig:closed-loop-1_ir}
  \end{figure}
  
\item
  CBR: Using a Constant Bit-Rate strategy, all frames need the same
  space. In the
  Fig.~\ref{fig:CBR} there is an
  example.
  \begin{figure}
    \svg{graphics/CBR}{500}
    \caption{Example of constant bit-allocation.}
    \label{CBR}
  \end{figure}
\end{itemize}

\section{Coding in the Transform Domain}

\subsection{The IPP... decorrelation pattern}
It's time to put together all the ``tools'' that we have developed for
encoding a sequence of frames $\{V_k\}$. First, the sequence will be
splitted into GOFs
(\href{https://en.wikipedia.org/wiki/Group_of_pictures}{Group Of
  Frames}), and the structure of each GOF will be
IPP...~\cite{le1991mpeg}, which means that the first frame of each GOF
will be intra-coded (I-type), and the rest of frames of the GOF will
be predicted-coded (P-type), respect to the previous one\footnote{A
  P-type frame except for the second one, that always has a I-frame as
  reference.}. Notice that in an I-type frame all the coefficients
(\emph{coeffs} in short, remember that we are compensating the motion
in the DWT domain) will be I-type coeffs, and in a P-type frame, the
different coeffs will be I-type or P-type.

\subsection{A block diagram of the step codec}

\begin{figure}
  \centering
  \svg{graphics/codec4}{1000}
  \caption{The IPP... MRVC step codec. Notice that the input to the
    (step) encoder is a DWT transformed sequence of frames.}
  \label{fig:codec}
\end{figure}

The MRVC IPP... step (one resolution level) codec has been described in the
Fig.~\ref{fig:codec}. The equations that describe this system are:

\begin{equation}
  (V_k.L, V_k.H) \leftarrow \text{DWT}(V_k),
  \tag{a}
\end{equation}
where $\leftarrow$ denotes the
\href{https://en.wikipedia.org/wiki/Assignment_(computer_science)}{assignment}
operator, and $V_k$ is the $k$-th frame of the sequence,

\begin{equation}
  [V_k.L] \leftarrow \text{DWT}^{-1}(V_k.L, 0),
  \tag{E.a}
\end{equation}

\begin{equation}
  Z^{-1}([V_k.L]) = [V_{k-1}.L],
  \tag{E.b}
\end{equation}
and by definition, $Z^{-1}([V_{-1}.L]) = 0$,

\begin{equation}
  \overset{k\rightarrow k-1}{V} \leftarrow \text{M}([V_k.L], [V_{k-1}.L]),
  \tag{E.c}
\end{equation}
where M stands for Motion estimation, and by definition,
$\overset{0\rightarrow (-1)}{V}=0$,

\begin{equation}
  [\hat{V}_k.L] \leftarrow \text{P}(\overset{k\rightarrow k-1}{V}, [V_{k-1}.L]),
  \tag{E.d}
\end{equation}
where P stands for motion compensated Prediction,

\begin{equation}
  [E_k.L] \leftarrow [V_k.L] - [\hat{V}_k.L],
  \tag{E.e}
\end{equation}

\begin{equation}
  \{[M_k],[S_k]\} \leftarrow \text{EW-min}([V_k.L], [E_k.L])
  \tag{E.f}
\end{equation}
where
\begin{equation}
  [M_k]_{i,j}=\text{min}([V_k.L]_{i,j}, [E_k.L]_{i,j}),
\end{equation}
and $[S_k]$ is a binary matrix defined by
\begin{equation}
  [S_k]_{i,j} = \left\{
  \begin{array}{lll}
    0 & \text{if}~[V_k.L]_{i,j} < [E_k.L]_{i,j} & \text{(I-type coeff)} \\
    1 & \text{otherwise}                      & \text{(P-type coeff)},
  \end{array}
  \right.
  \label{eq:matrix}
\end{equation}
(notice that $[M_k]$, that contains the element-wise minimum of both
matrices, is discarded)

\begin{equation}
  [V_k.H] \leftarrow \text{DWT}^{-1}(0, V_k.H),
  \tag{b}
\end{equation}

\begin{equation}
  [E_k.H] \leftarrow [V_k.H] - [\hat{V}_k.H],
  \tag{c}
\end{equation}
where, notice that
\begin{equation}
  [E_k.H]_{i,j} = \left\{
  \begin{array}{ll}
    {[}V_k.H{]}_{i,j}                       & \text{if}~{[}\hat{V}'_k.H{]}_{i,j} = 0~\text{(I-type coeff)} \\
    {[}V_k.H{]}_{i,j} - [\hat{V}'_k.H]_{i,j} & \text{otherwise}~\text{(P-type coeff)},
  \end{array}
\right.
\end{equation}

\begin{equation}
  [\tilde{E}_k.H] \leftarrow \text{Q}([E_K.H]),
  \tag{d}
\end{equation}

\begin{equation}
  [\tilde{E}_k.H] \leftarrow  \text{Q}^{-1}([\tilde{E}_K.H]),
  \tag{E.g}
\end{equation}

\begin{equation}
  [\tilde{V}_k.H] \leftarrow [\tilde{E}_k.H] + [\hat{V}'_k.H],
  \tag{E.h}
\end{equation}
and notice that if $[\hat{V}_k.H]=0$, then $[\tilde{V}_k.H] =
[\tilde{E}_k.H]$,

\begin{equation}
  Z^{-1}([\tilde{V}_k.H]) = [V_{k-1}.H],
  \tag{E.i}
\end{equation}
and by definition, $Z^{-1}([V_{-1}.H]) = 0$,

\begin{equation}
  [\hat{V}_k.H] \leftarrow \text{P}(\overset{k\rightarrow k-1}{V}, [\overset{\sim}{V}_{k-1}.H]),
  \tag{E.j}
\end{equation}

\begin{equation}
  [\hat{V}'_k.H]_{i,j} \leftarrow \left\{
    \begin{array}{ll}
      {[}\hat{V}_k.H{]}_{i,j}  & \text{if}~{[}E_k.L{]}_{i,j} < {[}V_k.L{]}_{i,j} \text{(P-type coeff)} \\
      0                       & \text{otherwise (I-type coeff)},
    \end{array}
  \right.
  \tag{E.k}
\end{equation}
  
\begin{equation}
  (0, \tilde{E}_k.H) \leftarrow \text{DWT}([\tilde{E}_k.H]),
  \tag{f}
\end{equation}

\begin{equation}
  \{V_k.L, \tilde{E}_k.H\} \leftarrow \text{E}(V_k.L, \tilde{E}_k.H),
  \tag{g}
\end{equation}
where E represents the entropy coding of both data sources, in two
different code-streams,

\begin{equation}
  (V_k.L, \tilde{E}_k.H) \leftarrow \text{E}^{-1}(\{V_k.L, \tilde{E}_k.H\}),
  \tag{h}
\end{equation}

\begin{equation}
  [\tilde{E}_k.H] \leftarrow \text{DWT}^{-1}(0, \tilde{E}_k.H),
  \tag{i}
\end{equation}

\begin{equation}
  (0, \tilde{V}_k.H) \leftarrow \text{DWT}(0, [\tilde{V}_k.H]),
  \tag{j}
\end{equation}

and

\begin{equation}
  \tilde{V}_k \leftarrow \text{DWT}^{-1}(V_k.L, \tilde{V}_k.H).
  \tag{k}
\end{equation}

The IPP... codec is inspired in Differential Pulse Code Moldulation.
This
\href{https://github.com/Sistemas-Multimedia/Sistemas-Multimedia.github.io/blob/master/milestones/12-IPP_coding/DPCM.ipynb}{notebook}
shows how to implement a simple DPCM codec.

\subsection{Spatial multiresolution}
As it can be seen in the previous section, in each IPP... iteration of
the step encoder, only the high-frequency information of the sequence
of frames is decorrelated ($H$ subbands) considering the information
provided by the low-frequences ($L$ subband), which are losslessly
transmitted between the encoder and the decoder. Notice also that if
the $L$ data cannot be transmitted to the decoder, a drift error will
occur because the matrix $S_k$ will be different in the encoder and
the decoder for the same frame $k$.

Obviously (and unfortunately), the lossless transmission of the $L$'s
bounds the compression ratio that we will get. One solution is to
perform more (than 1) levels at the DWT stage (see Eq.~(a)) and to
apply the IPP... MRVC step encoder by spatial resolutions, starting at
the lowest, as the decoder will do at decompression time. If we
represent the Spatial Resolution Level (SRL) with an superindex, being
0 the original SRL, we can express the operation of the codec
described in the Fig.~\ref{fig:codec} by
\begin{equation}
  \left\{
    \begin{array}{l}
      \text{SE}(V^0_k) = \{V^0_k.L, \tilde{E}^0_k.H\} = \{V^1_k, \tilde{E}^0_k.H\} \\
      \text{SD}(\{V^1_k, \tilde{E}^0_k.H\}) = \tilde{V}^0_k,
    \end{array}
  \right.
  \label{eq:codec_1l}
\end{equation}
where $\text{SE}(\cdot)$ represents to the operation of the
IPP... step encoder and $\text{SD}(\cdot)$ to the operation of the
IPP... step decoder. As it can be seen, Eq.~\ref{eq:codec_1l} is only
valid when only one level of the DWT has been applied.

In general, for $s$ levels of the DWT, we have that
\begin{equation}
  \left\{
    \begin{array}{l}
      \text{SE}(V^{s-1}_k) = \{V^s_k, \tilde{E}^{s-1}_k.H\} \\
      \text{SD}(\{V^s_k, \tilde{E}^{s-1}_k.H\}) = \tilde{V}^{s-1}_k,
    \end{array}
  \right.
  \label{eq:codec_sl}
\end{equation}
where $\tilde{V}^{s-1}$ is the $(s-1)$-th SRL of the reconstructed
sequence $\tilde{V}$.

The next SRL ($s-2$), $\tilde{V}^{s-2}$, is determined by
\begin{equation}
  \left\{
    \begin{array}{l}
      \text{SE}(\tilde{V}^{s-2}_k) = \{\tilde{V}^{s-1}_k, \tilde{E}^{s-2}_k.H\} \\
      \text{SD}(\{\tilde{V}^{s-1}_k, \tilde{E}^{s-2}_k.H\}) = \tilde{V}^{s-2}_k,
    \end{array}
  \right.
  \label{eq:codec_s1l}
\end{equation}
and finally, for the highest SRL, we get $\tilde{V}^0$ defined by
Eq.~\ref{eq:codec_1l}.
%\begin{equation}
%  \left\{
%    \begin{array}{l}
%      \text{C}(\tilde{V}^0_k) = \{\tilde{V}^1_k, \tilde{E}^0_k.H\} \\
%      \text{D}(\{\tilde{V}^1_k, \tilde{E}^0_k.H\}) = \tilde{V}^0_k.
%    \end{array}
%  \right.
%  \label{eq:codec_s1l}
%\end{equation}

\begin{figure}
  \centering
  \svg{graphics/encoder}{600}
  \caption{The IPP... MRVC encoder.}
  \label{fig:encoder}
\end{figure}

So, only the lowest SRL of $\tilde{V}$, $\tilde{V}^s$ is an
III... pure sequence of small frames, losslessly encoded. This
multiresolution (multistage) scheme has been described in the
Fig.~\ref{fig:encoder}. The output of an IPP... encoder will be
refered as ``spatial-layers'', or simply as ``S-layers''.

The Fig.~\ref{fig:decoder} shows the decoder. It inputs the collection
of subbands generated by the encoder and for each one, a video with a
different spatial resolution is obtained.

\begin{figure}
  \centering
  \svg{graphics/decoder}{320}
  \caption{The IPP... MRVC decoder.}
  \label{fig:decoder}
\end{figure}

\subsection{SQ (layers) progression (next milestone?)}
The logical order of the S-layers in the code-stream is the one that
allows, when the code-stream is decoded sequentially, the progressive
increase of the spatial resolution of the video. For example, if
$s$ is the number of levels of the DWT, the generated
code-stream has $(s+1)$ S-layers
\begin{equation*}
  \{V^s,\tilde{E}^{s-1}.H,\tilde{E}^{s-2}.H,\cdots,\tilde{E}^0.H\},
\end{equation*}
which are able to generate $(s+1)$ progressive
reconstructions
\begin{equation*}
  \{V^s,\tilde{V}^{s-1},\tilde{V}^{s-2},\cdots,\tilde{V}^0\}.
\end{equation*}

Moreover, Quality (Q-progression) scalability in each SRL can be also
achieved in the high-frequency textures if a quality-scalable image
codec such as JPEG2000~\cite{taubman2002jpeg2000} replaces the PNG
compressor, generating a number $q$ of quality-layers (``Q-layers'')
by each motion compensated high-frequency subband. A SQ-progression is
defined considering both forms of scalability (spatial and quality),
with a higher number of layers. For example, if $s=3$ (2 IPP...-type
iterations) and $q=2$, the progression of layers would be
\begin{equation*}
  \{V^s[1],V^s[0],\tilde{E}^{s-1}.H[1],\tilde{E}^{s-1}.H[0],\tilde{E}^{s-2}.H[1],\tilde{E}^{s-2}.H[0],\cdots,\tilde{E}^0.H[1],\tilde{E}^0.H[0]\}.
\end{equation*}
%Considering both forms of scalability (spatial and quality),
%the SQ-progression, that can be Pythonically-described by
%\begin{verbatim}
%progression = []
%for S_layer in range(N_SRL-1, 0, -1):
%  for Q_layer in range(N_QL-1, 0, -1):
%    progression.append((S_layer, Q_layer))
%\end{verbatim}

%For example, if $N_{\text{SRL}}=3$ (2 IPP...-type iterations) and
%$N_{\text{QL}}=2$, the progression of layers would be:
%\begin{verbatim}
%progression = [
%  (S_layer=2, Q_layer=1),  <-- First one to be decoded
%  (S_layer=2, Q_layer=0),
%  (S_layer=1, Q_layer=1),
%  (S_layer=1, Q_layer=0),
%  (S_layer=0, Q_layer=1),
%  (S_layer=0, Q_layer=0) ] <-- Last one
%\end{verbatim}

The use of quality scalability boosts the possibilities in real-time
streaming scenarios where the transmission bit-rate can be variable
(sending more or less Q-layers of a given spatial resolution depending
on the bit-rate). Notice that the SQ-progression is free of
drift-error.\footnote{Other progressions such as the QS-progression
  should generate drift because the decoder at some truncation points
  of the code-stream would use a low-frequency information with a
  different quality than the encoderd used.}

%However, notice that the rendering of the frames at
%the decoder side with a smaller quality (and this will depends on the
%transmission bit-rate) will generate a drift-error in the
%reconstruction of the video because the predictions used at the
%encoder and the decoder are not identical.

%When only spatial scalability is used, this scheme is free of
%drift-error.

\section{Linear frame interpolation using block-based motion compensation}
\label{sec:linear_frame_interpolation}
\begin{figure}[h]
  \svg{graphics/frame_interpolation}{900}
  \caption{Frame interpolation.}
  \label{fig:frame_interpolation}
\end{figure}

\subsection*{Input}
\begin{itemize}
\tightlist
\item
  $R$: square search area, in pixels.
\item
  $B$: square block size, in pixels.
\item
  $O$: border size, in pixels.
\item
  $s_i$, $s_j$ and $s_k$ three chronologically ordered,
  equidistant frames, with resolution $X\times Y$.
\item
  $A$: $\frac{1}{2^A}$ subpixel accuracy.
\end{itemize}

\subsection*{Output}
\begin{itemize}
\tightlist
\item
  $\hat{s}_j$: a prediction for frame $s_j$.
\item
  $m$: a matrix with $\lceil X/B\rceil \times \lceil Y/B\rceil$
  bidirectional motion vectors.
\item
  $e$: a matrix with $\lceil X/B\rceil \times \lceil Y/B\rceil$
  bidirectional Root Mean Square matching Wrrors (RMSE).
\end{itemize}

\subsection*{Algorithm}
\begin{enumerate}
\tightlist

\item
  Compute the DWT$^l$, where $l=\lfloor\log_2(R)\rfloor-1$ levels,
  of the predicted frame $s_j$ and the two reference frames $s_i$
  and $s_k$.
  \href{https://vicente-gonzalez-ruiz.github.io/video_compression/graphics/frame_interpolation_step_1.svg}{Example}.

\item
  $LL^l(m)\leftarrow 0$, or any other precomputed values (for example,
  from a previous ME in neighbor frames).
  \href{https://vicente-gonzalez-ruiz.github.io/video_compression/graphics/frame_interpolation_step_2.svg}{Example}.

\item
  Divide the subband $LL^l(s_j)$ into blocks of size $B\times B$
  pixels, and $\pm 1$-spiral-search them in the subbands $LL^l(s_i)$
  and $LL^l(s_k)$, calculating a low-resolution
  $LL^l(m)=\{LL^l(\overleftarrow{m}), LL^l(\overrightarrow{m})\}$
  bi-directional motion vector field. 
  \href{https://vicente-gonzalez-ruiz.github.io/video_compression/graphics/frame_interpolation_step_3A.svg}{Example}.
  \href{https://vicente-gonzalez-ruiz.github.io/video_compression/graphics/frame_interpolation_step_3A_bis.svg}{Example}.
\item
  While $l>0$:
  
  \begin{enumerate}
    
  \item
    Synthesize $LL^{l-1}(m)$, $LL^{l-1}(s_j)$, $LL^{l-1}(s_i)$
    and $LL^{l-1}(s_k)$, by computing the 1-level DWT$^{-1}$.
    \href{https://vicente-gonzalez-ruiz.github.io/video_compression/graphics/frame_interpolation_step_4A.svg}{Example}.
    \href{https://vicente-gonzalez-ruiz.github.io/video_compression/graphics/frame_interpolation_step_4A_bis.svg}{Example}

  \item
    $LL^{l-1}(M)\leftarrow LL^{l-1}(M)\times 2$.
    \href{https://vicente-gonzalez-ruiz.github.io/video_compression/graphics/frame_interpolation_step_4B.svg}{Example}.
  
  \item
    Refine $LL^{l-1}(m)$ using $\pm 1$-spiral-search.
    \href{https://vicente-gonzalez-ruiz.github.io/video_compression/graphics/frame_interpolation_step_4C.svg}{Example}.
  
  \item
  $l\leftarrow l-1$. (When $l=0$, the motion vectors field $m$
  has the structure:)
  
  \end{enumerate}
  
  \href{https://vicente-gonzalez-ruiz.github.io/video_compression/graphics/motion_vectors.svg}{Example}.
  
\item
  While $l<A$ (in the first iteration, $l=0$, and $LL^0(M):=M$):
  
  \begin{enumerate}
  \item
    $l\leftarrow l+1$.
    
  \item
    Synthesize $LL^{-l}(s_j)$, $LL^{-l}(s_i)$ and $LL^{-l}(s_k)$,
    computing the 1-level DWT$^{-1}$ (high-frequency subbands are
    $0$). This performs a zoom-in in these frames using $1/2$-subpixel
    accuracy. 
    
    \href{https://vicente-gonzalez-ruiz.github.io/video_compression/graphics/frame_interpolation_step_5B.svg}{Example}.
    
  \item
    $m\leftarrow m\times 2$.
    
    \href{https://vicente-gonzalez-ruiz.github.io/video_compression/graphics/motion_vectors_by_2.svg}{Example}.
    
  \item
    $B\leftarrow B\times 2$.
    
  \item
    Divide the subband $LL^{-l}(s_j)$ into blocks of $B\times B$
    pixels and $\pm 1$-spiral-search them into the subbands
    $LL^{-l}(s_i)$ and $LL^{-l}(s_k)$, calculating a $1/2^l$
    sub-pixel accuracy $m$ bi-directional motion vector field. 
    \href{https://vicente-gonzalez-ruiz.github.io/video_compression/graphics/motion_vectors_definitive.svg}{Example}.
    
  \item
    Frame prediction. For each block $b$:
    
  \item
    Compute
    \begin{equation}
      \hat{b}\leftarrow \frac{b_i\big(\overleftarrow{e}_{\text{max}}-\overleftarrow{e}(b)\big) + b_k\big(\overrightarrow{e}_{\text{max}}-\overrightarrow{e}(b)\big)}{\big(\overleftarrow{e}_{\text{max}}-\overleftarrow{e}(b)\big) + \big(\overrightarrow{e}_{\text{max}}-\overrightarrow{e}(b)\big)},
    \end{equation}
    
    where $\overleftarrow{e}(b)$ is the (minimum) distortion of the
    best backward matching for block $b$, $\overrightarrow{e}(b)$
    the (minimum) distortion of the best forward matching for block
    $b$,
    $\overleftarrow{e}_{\text{max}}=\overrightarrow{e}_{\text{max}}$ are
    the backward and forward maximum matching distortions, $b_i$ is
    the (backward) block found (as the most similar to $b$) in frame
    $s_i$ and $b_k$ is the (forward) block found in frame
    $s_k$. Notice that, if
    $\overleftarrow{e}(b)=\overrightarrow{e}(b)$, then the
    prediction is
    \begin{equation}
      \hat{b} = \frac{b_i + b_k}{2},
    \end{equation}
    and if $\overleftarrow{e}(b)=0$,
    \begin{equation}
      \hat{b} = b_k,
    \end{equation} and viceversa.
  \end{enumerate}
\end{enumerate}

\section{A problem}
\begin{figure}
  \centering
  \myfig{graphics/problem}{3cm}{300}
  \caption{Basic encoding problem.}
  \label{fig:problem}
\end{figure}

Using the encoding system described in the Figure~\ref{fig:problem}, and defined by
\begin{equation}
  \left\{\
    \begin{array}{l}
      \tilde{\mathbf R} = \text{Q}_{\mathbf R}({\mathbf R}) \\
      \tilde{\mathbf E} = \text{Q}_{\mathbf E}\big({\mathbf P}-\overset{{\mathbf R}\rightarrow {\mathbf P}}{\mathbf M}(\tilde{\mathbf R})\big)
    \end{array}
  \right.
  \label{eq:forward}
\end{equation}
and
\begin{equation}
  \begin{array}{l}
    \tilde{\mathbf P} = \tilde{\mathbf E} + \overset{{\mathbf R}\rightarrow {\mathbf P}}{\mathbf M}(\tilde{\mathbf R}),
  \end{array}
  \label{eq:backward}
\end{equation}

find $\text{Q}_{\mathbf{R}}$ and $\text{Q}_{\mathbf{E}}$ that minimize in the RD domain (the RD curve of)
\begin{equation}
  \text{MSE}(\{\mathbf{R},\mathbf{P}\},\{\hat{\mathbf{R}},\hat{\mathbf{P}}\}) = \frac{\text{MSE}({\mathbf R},\hat{\mathbf R}) + \text{MSE}({\mathbf P},\hat{\mathbf P})}{2},
\end{equation}
set that
\begin{equation}
  \text{MSE}({\mathbf R},\tilde{\mathbf R}) = \text{MSE}({\mathbf P},\tilde{\mathbf P}).
  \label{eq:constant_quality}
\end{equation}
Equation~\ref{eq:constant_quality} indicates that all the decoded
frames should have the same distortion (from a human perception point
of view). Notice that the transform defined by the Equations
~\ref{eq:forward} and \ref{eq:backward} is not orthogonal and
therefore, the ``subbands'' $\tilde{\mathbf R}$ and
$\tilde{\mathbf P}$ are not independent. It can be seen that
$\text{Q}_{\mathbf R}$ affects to the selection of
$\text{Q}_{\mathbf E}$, because $\tilde{\mathbf R}$ is used as
reference for finding ${\mathbf E}$.


\section{References}

\renewcommand{\addcontentsline}[3]{}% Remove functionality of \addcontentsline
\bibliography{image_pyramids,DWT,motion_estimation,HEVC,JPEG2000,video_compression}
